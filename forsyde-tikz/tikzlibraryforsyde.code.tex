% Author: George Ungureanu <ugeorge@kth.se>
% Date: 29.01.2017
%
% This file contains macros for drawing all nodes



%%%%%%%%%%%%%%%%%%%%%%%
% GLOBAL DECLARATIONS %
%%%%%%%%%%%%%%%%%%%%%%%

\pgfdeclarelayer{backbackground}
\pgfdeclarelayer{background}
\pgfdeclarelayer{foreground}
\pgfsetlayers{backbackground,background,main,foreground}

%%%%%%%%%%%
% HELPERS %
%%%%%%%%%%%

\newcommand\addcase[3]{\expandafter\def\csname\string#1@case@#2\endcsname{#3}}
\newcommand\makeswitch[2][]{%
  \newcommand#2[1]{%
    \ifcsname\string#2@case@##1\endcsname\csname\string#2@case@##1\endcsname\else#1\fi%
  }%
}

% \newcommand{\getmoclabel}[1]{%
%   \edef\mytemp{{#1}}%
%    \expandafter\ifstrequal\expandafter\mytemp{sy}{SY}{%
%      \expandafter\ifstrequal\expandafter\mytemp{de}{DE}{%
%        \expandafter\ifstrequal\expandafter\mytemp{ct}{CT}{%
%          \expandafter\ifstrequal\expandafter\mytemp{sdf}{SDF}{}}}}}
% \newcommand{\getmocfill}[1]{%
%    \ifstrequal{#1}{sy}{fill=sycolor}{%
%      \ifstrequal{#1}{de}{fill=decolor}{%
%        \ifstrequal{#1}{ct}{fill=ctcolor}{%
%          \ifstrequal{#1}{sdf}{fill=sdfcolor}{}}}}}

\makeswitch[]\getmoclabel
\addcase\getmoclabel{sy}{SY}
\addcase\getmoclabel{de}{DE}
\addcase\getmoclabel{ct}{CT}
\addcase\getmoclabel{sdf}{SDF}

\makeswitch[\defaultfillcolor]\getfieldcolor
\addcase\getfieldcolor{sy}{sycolor}
\addcase\getfieldcolor{de}{decolor}
\addcase\getfieldcolor{ct}{ctcolor}
\addcase\getfieldcolor{sdf}{sdfcolor}
\addcase\getfieldcolor{blackbox}{blackboxcolor}

\newcommand{\fillfield}[1][\@nil]{%
  \def\tmp{#1}%
  \ifx\tmp\@nnil
  \else\getfieldcolor{#1}
  \fi
}

\makeswitch[10pt]\getnodeysep
\addcase\getnodeysep{composite}{18pt}

\newcommand{\functionlabels}[5]{ %#1=nf, #2... = functions
  \ifnum #1>0 \nodepart{fa} #2\else\fi%
  \ifnum #1>1 \nodepart{fb} #3\else\fi%
  \ifnum #1>2 \nodepart{fc} #4\else\fi%
  \ifnum #1>3 \nodepart{fd} #5\else\fi%
}

%%%%%%%% 
% KEYS %
%%%%%%%%

\pgfkeys{/forsyde keys/.is family, /forsyde keys,
  %shape & color
  moc/.estore in         = \fsdMoc,
  type/.estore in        = \fsdType,
  inner shape/.estore in = \fsdInnerShape,
  outer shape/.estore in = \fsdOuterShape,
  shape/.style           = {outer shape = #1 },
  blackbox/.style        = {moc=blackbox},
  % separation & alignment
  anchor/.estore in     = \fsdAnchor,
  inner xsep/.estore in = \fsdInnerXSep,
  inner ysep/.estore in = \fsdInnerYSep,
  inner sep/.style      = {inner xsep = #1, inner ysep = #1 },
  % position
  xshift/.estore in = \fsdXShift,
  yshift/.estore in = \fsdYShift,
  at/.estore in     = \fsdAtInit,
  left of/.style  = {at={#1.east}, xshift=1cm, anchor=west},
  right of/.style = {at={#1.west}, xshift=-1cm, anchor=east},
  above of/.style = {at={#1.north}, yshift=1cm, anchor=south},
  below of/.style = {at={#1.south}, yshift=-1cm, anchor=north},
  % rotation
  rotate inner/.estore in = \fsdRotateInner,
  rotate outer/.estore in = \fsdRotateOuter,
  rotate/.estore in       = \fsdRotate,
  reverse inner/.is toggle,
  reverse outer/.is toggle,
  reverse/.style          = {reverse outer, reverse inner },
  % ports & functions
  npl/.estore in = \fsdNPortLeft,
  npr/.estore in = \fsdNPortRight,
  nf/.estore in  = \fsdNFunctions,
  f1/.estore in  = \fsdFunA,
  f2/.estore in  = \fsdFunB,
  f3/.estore in  = \fsdFunC,
  f4/.estore in  = \fsdFunD,
  ni/.style      = {npl = #1 },
  no/.style      = {npr = #1 },
  f/.style = {f1 = {#1}, nf = 1, },
  % default values
  default/.style = {
    moc           = none,
    type          =,
    inner shape   = rectangle,
    outer shape   = rectangle,
    anchor        = center,
    inner ysep    = 10pt,
    inner xsep    = 10pt,
    reverse inner = false, 
    reverse outer = false,
    rotate        = 0,
    rotate inner  = \iftoggle{/forsyde keys/reverse inner}{180}{0},
    rotate outer  = \iftoggle{/forsyde keys/reverse outer}{180}{0},
    xshift        = 0pt,
    yshift        = 0pt,
    at            = {0,0},
    npl=1, npr=1, nf=0, 
    f1=$ f_1 $, f2=$ f_2 $, f3=$ f_3 $, f4=$ f_4 $,
  }
}

%%%%%%%%%%%%%%%%%%%%%%%%%%%
% SHAPES OF MAIN ELEMENTS %
%%%%%%%%%%%%%%%%%%%%%%%%%%%

\tikzset{
  nary/.style     ={double distance=1pt, outer sep=.6pt},
  primitive/.style={draw,circle,inner sep=0pt, minimum size=15pt, fill=\defaultfillcolor},
  leaf/.style     ={draw, rectangle, rounded corners=3pt, inner sep=0pt},
  composite/.style={draw, rectangle, dashed, fill opacity=0},
}


\ExplSyntaxOn
% regex function to strip the "|" delimiter between function
% arguments.
\tl_new:N \l_stripdelim_tl
\cs_new:Npn \stripdelim #1 {
  \tl_set:Nn \l_stripdelim_tl {#1}
  \regex_replace_all:nnN { | } { } \l_stripdelim_tl
  \tl_use:N \l_stripdelim_tl
}

\NewDocumentCommand{\makeFunctions}{m m} {%
  \def\dummylbl{\funcStyle\stripdelim{#2}}
  \def\parentName{#1}
  \coordinate (tmpPosOfFunBox) at (0,0);
  \pgfmathtruncatemacro{\countFun}{0};
  % \begin{pgfonlayer}{foreground}
    \drawFunBoxes{ #2 }
  % \end{pgfonlayer}
}
\NewDocumentCommand{\drawFunBoxes}{ >{ \SplitList { | } } m } {
  \ProcessList { #1 } { \davs__tokens_drawfun_rec:n }
}   
\cs_new_protected:Nn \davs__tokens_drawfun_rec:n {
  \pgfmathtruncatemacro{\countFun}{\countFun + 1}
  \node[draw, thin, fill=\defaultfillcolor, anchor=west,
    inner\space sep=2pt] (\parentName\countFun) at (tmpPosOfFunBox)
    {\vphantom{\dummylbl}\funcStyle #1};
  \coordinate[xshift=2pt]
    (tmpPosOfFunBox) at (\parentName\countFun.east);   
}
\ExplSyntaxOff




%%%%%%%%%%%%%%%%%%%%% 
% MACRO DEFINITIONS %
%%%%%%%%%%%%%%%%%%%%%

% The basic node is the simplest depiction of a function (atom,
% pattern) with maximum one argument. It is represented as a circle
% node.
% #1 = keys
% #2 = name
% #3 = position
% #4 = label
\newcommand{\basic}[4][]{
  \pgfkeys{/forsyde keys, default, #1}%
  \node[\fsdOuterShape] (#2) at (#3) {#4};
  \begin{pgfonlayer}{foreground}
    \node[func\fsdNFunctions, font=\funcStyle](#2-f) at (#2.north){
      \functionlabels{\fsdNFunctions}{\fsdFunA}{}{}{}};
  \end{pgfonlayer}   
}


% The standard node is a more complex depiction of a ForSyDe block. It
% contains anchors for each port, shows a label and the constructor
% name, and its field is colored according to a MoC (if this is the
% case).
%
% \standard[keys] (name) [<position>] {label};
\NewDocumentCommand\standard{O{} r() D<>{0,0} m}{
  \pgfkeys{/forsyde keys, default, #1}%
  \coordinate (globalPos) at (#3);
  % the scope binds together all other elements and positions them
  % according to the global coordinates. It also contains port
  % anchors.
  \begin{scope}[%
    ports i\fsdNPortLeft o\fsdNPortRight,
    local bounding box=#2,
    shift={($(globalPos) + (\fsdAtInit)$)},
    anchor=\fsdAnchor,
    rotate=\fsdRotate]%
    % a label should be in the center, controlled with \labelStyle
    \node[] (#2-l) at (0,0) {\labelStyle{#4}};
    % the functions are inside the scope, above the label
    \node[%
      func\fsdNFunctions,
      font=\funcStyle,
      yshift=\fsdInnerYSep,
      anchor=north] (#2-f) at (#2-l.north) {%
        \functionlabels{\fsdNFunctions}%
        {\fsdFunA}{\fsdFunB}{\fsdFunC}{\fsdFunD}};
    % the type (constructor) is positioned below the label
    \node[%
      yshift=-\fsdInnerYSep,
      inner sep=2pt,
      anchor=south] (#2-t) at (#2-l.south){%
        \typeStyle{\fsdType \ifnolabel\else\getmoclabel{\fsdMoc}\fi}};
    % The main shape is behind any other graphical element.
    \begin{pgfonlayer}{background}
      \node[\fsdOuterShape,
        fill=\getfieldcolor{\fsdMoc},
        rotate=\fsdRotateOuter,
        fit=(#2-f)(#2-l)(#2-t)] {};
    \end{pgfonlayer}
  \end{scope}
}


% Custom leaf node. It is the least flexible and contains no
% additional info nor labels. It accepts any shape or drawing as
% `outer shape` and it binds ports to inputs.
% #1 = keys
% #2 = name
% #3 = position
\newcommand{\drawing}[3][]{
  \pgfkeys{/forsyde keys, default, #1}%
  % artifice to fix the inner ysep according to the maximum number of
  % ports
  \pgfmathsetlength{\foo}{max(max(\fsdNPortLeft,\fsdNPortRight)*5pt,
    \fsdInnerYSep)}
  % the scope has only ports and contains the shape node
  \begin{scope}[%
    ports i\fsdNPortLeft o\fsdNPortRight,
    local bounding box= #2, shift={(#3)},
    anchor=\fsdAnchor,
    rotate=\fsdRotate]%
    % shape node
    \node[\fsdOuterShape,
      inner sep=\foo, 
      fill=\getfieldcolor{\fsdMoc},
      rotate=\fsdRotateOuter] {};
  \end{scope}
}

% Cluster node. Instead of taking a coordinate, it takes a list of
% nodes and draws the shape around them.
% #1 = keys
% #2 = name
% #3 = list of nodes clustered
% #4 = label
\newcommand\cluster[4][]{
  \pgfkeys{/forsyde keys, default, #1}%
  % the scope now contains only the cluster shape and the
  % labels. Functions are outside of it.
  \begin{scope}[%
    ports i\fsdNPortLeft o\fsdNPortRight,
    local bounding box= #2,
    rotate=\fsdRotate]%
    % cluster shape
    \begin{pgfonlayer}{backbackground}
    \node[\fsdOuterShape,
      inner xsep=\fsdInnerXSep,
      inner ysep=\fsdInnerYSep, 
      fill=\getfieldcolor{\fsdMoc},
      fit=#3, rotate=\fsdRotateOuter] (#2-s) {};
    \end{pgfonlayer}
    % label
    \node[anchor=south east] (#2-l) at (#2-s.south east){
      \footnotesize\labelStyle{#4}};
    % constructor  
    \node[anchor=south west] (#2-t) at (#2-s.south west){
      \typeStyle{\fsdType}}; 
  \end{scope}
  % the functions are in foreground, outside the scope of the cluster.
  \begin{pgfonlayer}{foreground}
    \node[func\fsdNFunctions,
    font=\funcStyle,
    anchor=south west] (#2-f) at (#2-s.north west) {%
      \functionlabels{\fsdNFunctions}%
      {\fsdFunA}{\fsdFunB}{\fsdFunC}{\fsdFunD}};
  \end{pgfonlayer}
}


% Interface
\newcommand{\interface}[3][]{
  \pgfkeys{/interface keys, default, #1}%
  \pgfmathtruncatemacro\rot{\iftoggle{/interface keys/reverse}{180}{0}}	
  \begin{scope}[local bounding box= #2, shift={(#3)}]
    \node[interface shape \rot, mocin=\pMocIn, mocout=\pMocOut, minimum width=30pt, minimum height=30pt, draw] (#2-shape) at (0,0) {
      \nodepart{mocin}  \typeStyle{\ifnolabel\else\getmoclabel{\pMocIn}\fi}
      \nodepart{mocout} \typeStyle{\ifnolabel\else\getmoclabel{\pMocOut}\fi} 
    };
    \node[ports i1o1, rotate=\rot, inner sep=0pt, fit=(#2-shape),] (#2-p) {};
  \end{scope}
}

% Embedding (hybrid) processes
% #1 = keys
% #2 = name
% #3 = list of nodes clustered
% #4 = label
\newcommand\embed[4][]{
  \pgfkeys{/embedded moc keys, default, #1}%
  \pgfmathtruncatemacro\rot{\iftoggle{/standard leaf keys/reverse}{180}{0}}
  \begin{scope}[on background layer, local bounding box= #2]
    \node[leaf shape, draw, inner xsep=\innerXSep, inner ysep=\innerYSep, moc=\pMoc,
    fit=#3] (#2-shape) {};
    \node[anchor=south east] (#2-label) at (#2-shape.south east) {\footnotesize\labelStyle{#4}}; 
    \node[ports i\pNIn o\pNOut, rotate=\rot, inner sep=0pt, fit=(#2-shape),] (#2-p) {};
  \end{scope}
}


%%%%%%%%%%%%%%%%%%%%%%%
% COMPOSITE PROCESSES %
%%%%%%%%%%%%%%%%%%%%%%%

%Composite

% Generic composite process
% #1 = keys
% #2 = name
% #3 = list of nodes clustered
% #4 = label
\newcommand\compositestd[4][]{
	\pgfkeys{/composite keys, default, #1}%
	\pgfmathtruncatemacro\rot{\iftoggle{/composite keys/reverse}{180}{0}}
	\begin{scope}[local bounding box= #2]
		\node[composite shape, inner xsep=\innerXSep, inner ysep=\innerYSep, fit=#3 , draw] (#2-shape) {};
		\node[anchor=south east] (#2-label) at (#2-shape.south east) {\footnotesize\labelStyle{#4}}; 
		\node[ports i\pNIn o\pNOut, rotate=\rot, inner sep=0pt, fit=(#2-shape),] (#2-p) {};
	\end{scope}
}
% Generic black-box composite process
% #1 = keys
% #2 = name
% #3 = position
% #4 = label
\newcommand\compositebbox[4][]{
	\pgfkeys{/composite keys, default, #1}%
	\pgfmathtruncatemacro\rot{\iftoggle{/composite keys/reverse}{180}{0}}
	\begin{scope}[local bounding box= #2]
		\node[rectangle, rounded corners = 3pt, minimum width=30pt, minimum height=30pt, inner xsep=\innerXSep,
			inner ysep=\innerYSep, draw=\defaultdrawcolor, fill=blackboxcolor] (#2-shape) at (#3) {\labelStyle{#4}};
		\node[ports i\pNIn o\pNOut, rotate=\rot, inner sep=0pt, fit=(#2-shape),] (#2-p) {};
	\end{scope}
}


%%%%%%%%%%%%
% PATTERNS %
%%%%%%%%%%%%

% Applicative pattern
% #1 = keys
% #2 = name
% #3 = list of nodes clustered
% #4 = label
\newcommand\patterncluster[4][]{
	\pgfkeys{/applicative patterns keys, default, #1}%
	\pgfmathtruncatemacro\rotports{\iftoggle{/applicative patterns keys/reverse}{180}{0}}
	\pgfmathtruncatemacro\rotshape{\iftoggle{/applicative patterns keys/reverse shape}
		{\iftoggle{/applicative patterns keys/reverse}{0}{180}}
		{\iftoggle{/applicative patterns keys/reverse}{180}{0}}}
	\begin{scope}[local bounding box= #2]
		\node[\pShape, draw, inner xsep=\innerXSep, inner ysep=\innerYSep, fit=#3, rotate=\rotshape]
			(#2-shape) {};
		\node[inner xsep=\innerXSep, inner ysep=\innerYSep, fit=#3] (#2-scope) {};
		\node[ports i\pNIn o\pNOut, rotate=\rotports, inner sep=0pt, fit=(#2-shape),]  (#2-p) {};	
		\node[anchor=south east] (#2-label) at (#2-scope.south east) {\footnotesize\labelStyle{#4}}; 
		\node[anchor=south west] (#2-type)  at (#2-scope.south west) {\typeStyle{\pType}}; 
		\node[func\pNFunc, font=\funcStyle, anchor=south west] (#2-f) at (#2-scope.north west) 
			{\functionlabels{\pNFunc}{\pFuncA}{\pFuncB}{\pFuncC}{\pFuncD}};
	\end{scope}
}
% Standard transition pattern
% #1 = keys
% #2 = name
% #3 = position
\newcommand\patternnodestd[3][]{
	\pgfkeys{/transition patterns keys, default, #1}%
	\pgfmathtruncatemacro\rotports{\iftoggle{/transition patterns keys/reverse}{180}{0}}
	\pgfmathtruncatemacro\rotshape{\iftoggle{/transition patterns keys/reverse shape}{180}{0}}
	\begin{scope}[local bounding box= #2, shift={(#3)}]	
		\begin{scope}	
			\node[rotate=90] (#2-type) at (0,0) {\typeStyle{\pType}};
			\node[fit=(#2-type), inner sep=0pt](#2-wt) {};
			\node[func\pNFunc, font=\funcStyle, anchor=south, inner sep=0pt] (#2-f) at (#2-wt.north) 
				{\functionlabels{\pNFunc}{\pFuncA}{\pFuncB}{\pFuncC}{\pFuncD}};
			\node[\pOuterShape, draw, inner sep=0pt, fit=(#2-wt), rotate=\rotshape,] (#2-shape) {};
			\node[ports i\pNIn o\pNOut, rotate=\rotports, inner sep=0pt, fit=(#2-shape),] (#2-p) {};
		\end{scope}
	\end{scope}
}
% Custom transition pattern
% #1 = keys
% #2 = name
% #3 = position
\newcommand\patternnodecustom[3][]{
	\pgfkeys{/transition patterns keys, default, #1}%
	\pgfmathtruncatemacro\rotports{\iftoggle{/transition patterns keys/reverse}{180}{0}}
	\pgfmathtruncatemacro\rotshape{\iftoggle{/transition patterns keys/reverse shape}
		{\iftoggle{/transition patterns keys/reverse}{0}{180}}
		{\iftoggle{/transition patterns keys/reverse}{180}{0}}}
	\begin{scope}[local bounding box= #2, shift={(#3)}]	
		\begin{scope}	
			\node[\pInnerShape, rotate=\rotports] (#2-c) at (0,0) {};
			\node[func\pNFunc, font=\funcStyle, anchor=south] (#2-f) at (#2-c.north) 
				{\functionlabels{\pNFunc}{\pFuncA}{\pFuncB}{\pFuncC}{\pFuncD}};
			\node[\pOuterShape, draw, inner xsep=0pt, inner ysep=1pt, fit=(#2-c), rotate=\rotshape,] 
				(#2-oshape) {};
			\node[ports i\pNIn o\pNOut, rotate=\rotports, inner sep=0pt, fit=(#2-oshape),] (#2-p) {};
		\end{scope}
	\end{scope}
}

%%%%%%%%%
% PATHS %
%%%%%%%%%
% \def\signal[#1] (#2) #3 (#4);{
%         \draw (#2) edge[#1, #3, s=\MoC,] (#4);
% }
% \def\vector[#1] (#2) #3 (#4);{
%         \draw (#2) edge[#1, #3, v=\MoC,] (#4);
% }
% \def\function[#1] (#2) #3 (#4);{
%         \draw (#2) edge[#1, #3, f,] (#4);
% }
