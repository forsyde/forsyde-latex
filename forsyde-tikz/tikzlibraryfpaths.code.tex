% pgf/tikz library
% for ForSyDe signals
%
% Author: George Ungureanu, KTH - Royal Institute of Technology, Sweden
% Version: 0.3
% Date: 2015/05/17

%

%
% Styles for signal, vector and function paths.
%
\tikzset{
  getsignalmoc/.code={%
    \ifnocolor \defaultdrawcolor \else
    \ifstrequal{#1}{sy}{\tikzset{sycolor}}{%
      \ifstrequal{#1}{de}{\tikzset{decolor}}{%
        \ifstrequal{#1}{ct}{\tikzset{ctcolor}}{%
          \ifstrequal{#1}{sdf}{\tikzset{sdfcolor}}{}}}}
    \fi
  },
  s/.style args  = {#1}{getsignalmoc={#1}, line width=\signalpathlinewidth, >= stealth},
  sn/.style args = {#1}{getsignalmoc={#1}, line width=.5,  >= stealth, double distance=1.3pt},
  v/.style args  = {#1}{getsignalmoc={#1}, line width=\vectorpathlinewidth, >= stealth},
  vn/.style args = {#1}{getsignalmoc={#1}, line width=2, double distance=2pt, >= stealth, double},
  f/.style       = {line width=\functionpathlinewidth, >= open triangle 90, dotted},
  fn/.style      = {line width=.4, >= open triangle 90, dotted, double distance=1pt, double},
  srcport/.style={%
    shorten <=2pt,%
    decoration={%
      name=markings,
      mark=at position 0 with {\node[fill=\defaultdrawcolor,inner sep=2pt]{};}%
    },%
    postaction=decorate,
  },
  dstport/.style={%
    shorten >=2pt,%
    decoration={%
      name=markings,
      mark=at position 1 with {\node[fill=\defaultdrawcolor,inner sep=2pt]{};}%
    },%
    postaction=decorate,
  },
}

\tikzset{
  trans/.style n args={3}{%
    decoration={show path construction, lineto code={
      \path[draw opacity=0, name path=this path] (\tikzinputsegmentfirst) -- (\tikzinputsegmentlast); 
      \path[name intersections={of=this path and #2,by={int-1}}];%
      \path (\tikzinputsegmentfirst) edge[#1] (int-1)
            (int-1) edge[#3] (\tikzinputsegmentlast); 
      }
    },
    decorate,
  },
  trans2/.style n args={5}{%
    decoration={show path construction, lineto code={
      \path[draw opacity=0, name path=this path] (\tikzinputsegmentfirst) -- (\tikzinputsegmentlast); 
      \path[name intersections={of=this path and #2,by={first-int}}];% 
      \path[name intersections={of=this path and #4,by={second-int}}];%
      \path (second-int) edge[#5] (\tikzinputsegmentlast)
            (first-int)  edge[#3] (second-int)
            (\tikzinputsegmentfirst) edge[#1] (first-int);
      }
    },
    decorate,
  },
}


%
% Styles for data tokens.
%

\newlength{\xTemp}
\newlength{\inTemp}
\newlength{\lTemp}
\newlength{\startFunc}
\ExplSyntaxOn

\NewDocumentCommand{\tokens}{m}
 {% #1 = character, #2 = string
  \youra_count_char:nn { - } { #1 }
  \drawtokens{ #1 }
 }
\int_new:N \l_youra_count_char_int
\cs_new_protected:Npn \youra_count_char:nn #1 #2
 {
  \regex_count:nnN { #1 } { #2 } \l_youra_count_char_int
 }
 
\NewDocumentCommand{\drawtokens}{ >{ \SplitList { - } } m }
{
 \pgfmathsetlength{\xTemp}{-\l_youra_count_char_int * \tokensize * \globScale}
 \pgfmathsetlength{\lTemp}{\tokensize + 0.3pt}
 \setlength{\inTemp}{\halftokensize}
 \ProcessList { #1 } { \davs__tokens_drawbox_rec:n }
 \pgfmathsetlength{\xTemp}{-\l_youra_count_char_int * \tokensize * \globScale}
 \setlength{\inTemp}{\halftokensize}
 \ProcessList { #1 } { \davs__tokens_drawtok_rec:n }
}

\cs_new_protected:Nn \davs__tokens_drawbox_rec:n
 {
  \IfSubStr { #1 } { ( }{ 
    \pgfmathsetlength{\startFunc}{ \xTemp - 0.3pt - \tokensize * \globScale  } 
  }{}
  \IfSubStr { #1 } { ) }{ 
    \pgfmathsetlength{\inTemp}{ \xTemp + 0.3pt + \tokensize * \globScale }
    \pgfsetlinewidth{0.2pt}
    \draw[ fill=\defaultfillcolor] (\startFunc,-\lTemp) rectangle (\inTemp,\lTemp); 
  }{}
  \pgfmathsetlength{\xTemp}{\xTemp+ 2 * \globScale * \tokensize}
 }

\cs_new_protected:Nn \davs__tokens_drawtok_rec:n
 {
  \davs__tokens_do:n { #1 } 
  \pgfmathsetlength{\xTemp}{\xTemp+ 2 * \globScale * \tokensize}
 }

\cs_new_protected:Nn \davs__tokens_do:n
 {
	\IfSubStr { #1 } { scalar }{
		\node[scalartokenshape, draw] at (\xTemp,0) {};
	}{
  	\IfSubStr { #1 } { vector }{
		\node[vectortokenshape, draw] at (\xTemp,0) {};
	}{
  	\IfSubStr { #1 } { function }{
		\node[functiontokenshape, draw] at (\xTemp,0) {};
	}{}
	}}
 }

\ExplSyntaxOff


%
% Path splines syntax
%
\tikzset{
  token/.style={decoration={name=markings, post length=1pt, pre length=1pt, mark=at position \tokPos with {\tokens{#1}}}, postaction=decorate },
  -|/.style={ to path={ (\tikztostart) -| (\tikztotarget) \tikztonodes }},
  |-/.style={ to path={ (\tikztostart) |- (\tikztotarget) \tikztonodes }},
  -|-/.style={ to path={ (\tikztostart) 
	-| ([xshift=\pDeviate]$(\tikztostart)!#1!(\tikztotarget)$) 
	|- (\tikztotarget) 
	\tikztonodes }
  }, -|-/.default=0.5,
  |-|/.style={to path={ (\tikztostart) 
	|- ($(\tikztostart)!#1!(\tikztotarget)$) 
	-| (\tikztotarget)
    \tikztonodes }
  }, |-|/.default=0.5,
  -|-|/.style args = {#1}{ to path={ (\tikztostart) 
	-| ($(\tikztostart)!#1!(\tikztotarget)$) 
	|- ([yshift=\pDeviate]\tikztotarget) 
	-| (\tikztotarget) 
	\tikztonodes }
  }, -|-|/.default=0.5,
  -|-|-/.style args = {#1:#2}{ to path={ (\tikztostart) 
	-| ($(\tikztostart)!#1!(\tikztotarget)$) 
	|- ([yshift=\pDeviate]$(\tikztostart)!#2!(\tikztotarget)$) 
	-| ([yshift=\pDeviate]$(\tikztostart)!#2!(\tikztotarget)$)
	|- (\tikztotarget) 
	\tikztonodes }
  }, -|-|-/.default=0.3:0.7,
  |-|-|/.style args = {#1:#2}{ to path={ (\tikztostart) 
	|- ($(\tikztostart)!#1!(\tikztotarget)$) 
	-| ([xshift=\pDeviate]$(\tikztostart)!#2!(\tikztotarget)$) 
	|- ([xshift=\pDeviate]$(\tikztostart)!#2!(\tikztotarget)$)
	-| (\tikztotarget) 
	\tikztonodes }
  }, |-|-|/.default=0.3:0.7,
}

% Finally, edge drawing helpers
\def\signal[#1] (#2) #3 (#4);{
        \draw (#2) edge[#1, #3, s=\MoC,] (#4);
}
\def\signaln[#1] (#2) #3 (#4);{
        \draw (#2) edge[#1, #3, sn=\MoC,] (#4);
}
\def\vector[#1] (#2) #3 (#4);{
        \draw (#2) edge[#1, #3, v=\MoC,] (#4);
}
\def\vectorn[#1] (#2) #3 (#4);{
        \draw (#2) edge[#1, #3, vn=\MoC,] (#4);
}
\def\function[#1] (#2) #3 (#4);{
        \draw (#2) edge[#1, #3, f,] (#4);
}
\def\functionn[#1] (#2) #3 (#4);{
        \draw (#2) edge[#1, #3, fn,] (#4);
}

\endinput
