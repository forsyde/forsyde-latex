\documentclass[10pt]{article}
\usepackage[a4paper]{geometry}
\usepackage[footnotesize]{caption}
\usepackage{longtable}
\usepackage{booktabs}
\usepackage{hyperref}
\usepackage{enumitem}
\usepackage{marginnote}
\usepackage{listings}
\usepackage[tikz,math]{forsyde}

\title{The \ForSyDeLaTeX\ utilities}
\author{
  George Ungureanu \\
  Department of Electronic Systems\\
  KTH Royal Institute of Technology\\
  Stockholm, SWEDEN
}
\date{\today}



\newenvironment{optionslist}[0]{ 
\begin{list}{}{
	\setlength{\itemindent}{-10pt}
%	\setlength{\topsep}{0pt}
	\setlength{\itemsep}{0pt}
	\setlength{\parsep}{0pt}
}}{\end{list}}
\newcommand\bookmark[1]{\marginpar{\ttfamily #1}}
\lstset{
  basicstyle=\footnotesize\ttfamily,
  % numbers=left,
  frame=single,
  numberstyle=\tiny\color{black!30},
  commentstyle=\color{blue}\textit,
  stringstyle=\color{magenta}\textit,
  flexiblecolumns=false,
  basewidth={0.5em,0.45em},
  breaklines=true,
  language={[LaTeX]TeX},
  texcsstyle=*\color{red}\bfseries,
  keywordstyle=\color{blue}\bfseries,
  morekeywords={tikzpicture,document},
  moretexcs={trans,standard,interface,basic,cluster,node,path,embed},
}
\def\opt#1{\color{gray}{#1}}
\def\man#1{\color{black}{#1}}

\begin{document}
\maketitle
\reversemarginpar

\begin{abstract}
This is the reference manual for the \LaTeX\ utilities used in the context of \ForSyDe. All packages and their API features are documented here.
\end{abstract}

\section{Introduction}

This library was developed as an effort to standardize symbols and graphical primitives in documents related to \ForSyDe, but also to provide tools and utilities for user convenience. \ForSyDe is a high-level design methodology aiming at synthesizing correct-by-construction systems through formal means. For more information check \url{https://forsyde.ict.kth.se}.

The library contains the following main packages:
\begin{itemize}
\item \texttt{forsyde-tikz} : is a collection of \textsc{PGF} and \textsc{TikZ} styles, graphical primitives and commands for drawing \ForSyDe process networks;
\item \texttt{forsyde-math} : is a collection of math symbols used in the \ForSyDe formal notation. It is mainly focused on the ongoing \ForSyDeAtom methodology;
\item \texttt{forsyde-plot} : provides utilities for plotting \ForSyDe signals;
\item \texttt{forsyde-legacy} : API for the previous versions of this library.
\end{itemize}

\section{Installation \& usage}

There are three main alternatives to install the libraries:

\begin{enumerate}
\item copy the contents of \texttt{forsyde-latex/src} and \texttt{forsyde-latex/fonts} in their appropriate path under \texttt{TEXMFHOME} or any standard loading path, as specified by your \LaTeX\ compiler. Refer to \url{https://en.wikibooks.org/wiki/LaTeX/Installing_Extra_Packages} for more information.
\item compile your document with the variable \texttt{TEXINPUTS} set to \texttt{/path/to/forsyde-latex/src/}. If you intend to use \texttt{forsyde-math} characters, you need to generate the fonts under \texttt{forsyde-latex/fonts} using a \texttt{METAFONT} tool suite, and afterwards compile your document with the variable \texttt{TEXFONTS} set to \texttt{/path/to/forsyde-latex/fonts/};
\item copy the contents of \texttt{forsyde-latex/src} and \texttt{forsyde-latex/fonts} in the same folder as your document and compile normally.
\end{enumerate}

To include any of the packages enumerated in the introduction, you cal load the \texttt{forsyde} package with the appropriate option:

\begin{verbatim}
	\usepackage[option]{forsyde}
\end{verbatim}
where \texttt{option} is
\begin{itemize}
\item \texttt{tikz} for loading the \texttt{forsyde-tikz} library
\item \texttt{math} for loading the \texttt{forsyde-math} library
\item \texttt{plot} for loading the \texttt{forsyde-plot} library
\item \texttt{legacy} for loading the \texttt{forsyde-legacy} library
\end{itemize}

When loaded without an option, this package only provides some general commands for typesetting and logos:

\begin{longtable} { c | c }
  \toprule
  \textbf{Command}  & \textbf{Expands to} \\
  \midrule
  \texttt{\string\ForSyDe}      & \ForSyDe \\
  \texttt{\string\ForSyDeLaTeX} & \ForSyDeLaTeX \\
  \texttt{\string\ForSyDeAtom} & \ForSyDeAtom \\
  \bottomrule
\end{longtable}

\newpage
\tableofcontents
\newpage
% A simple library for signal flow diagrams
% based on the pgf/tikz package of Till Tantau
%
% Author: Dr. Karlheinz Ochs, Ruhr-University of Bochum, Germany
% Version: 0.1
% Date: 2007/01/05
\NeedsTeXFormat{LaTeX2e}
\RequirePackage{pgfplots}
\RequirePackage{pgfkeys}
\RequirePackage{xparse}
\RequirePackage{todonotes} % WHA?!
\RequirePackage{ezkeys}
\usetikzlibrary{decorations.markings, shapes, calc, fit, backgrounds}

\ProvidesPackage{forsyde-tikz}
              [2014/12/01 v0.1 ForSyDe TikZ Library]

%
% Libraries for signal flow diagrams.
%
\usetikzlibrary{fsignals,fshapes,}


%%%%%%%%%%%%%
% CONSTANTS %
%%%%%%%%%%%%%
% Colors
\newcommand{\defaultdrawcolor}{black}     		% draw color of signal paths
\newcommand{\defaultfillcolor}{white}     		% draw color of signal paths
\definecolor{sycolor}{RGB}{148,183,215}
\definecolor{ctcolor}{RGB}{225,119,19}
\definecolor{decolor}{RGB}{80,229,154}
\definecolor{sdfcolor}{RGB}{220,220,20}
\definecolor{blackboxcolor}{gray}{0.80}
% line widths of
\newlength{\sepq}
\pgfmathsetlength{\sepq}{2pt}                % constant for small inter-node separation (used for example in function nodes)
\newcommand{\compositelinewidth}{.4pt}       % composite process line width
\newcommand{\skeletonlinewidth}{1pt}         % parallel processes line width
\newcommand{\signalpathlinewidth}{1pt}       % signal paths
\newcommand{\functionpathlinewidth}{.8pt}    % function paths
\newcommand{\vectorpathlinewidth}{4pt}       % vector paths
% sizes, etc.
\newcommand{\tokensize}{3.5pt}
\newcommand{\halftokensize}{1.75pt}
\newcommand{\vectorportsize}{3pt}
\newcommand{\signalportsize}{2pt}

%%%%%%%%%%%%%%%%%%%%%%%%
% GENERIC TIKZ HELPERS %
%%%%%%%%%%%%%%%%%%%%%%%%
% Positioning of node text.
% #1 = node label
% #2 = label text
\newcommand{\textaboveof}[2]{\pgftext[bottom,at=\pgfpointanchor{#1}{north},y=+1mm]{#2}}%
\newcommand{\textrightof}[2]{\pgftext[left,  at=\pgfpointanchor{#1}{east}, x=+1mm]{#2}}%
\newcommand{\textbelowof}[2]{\pgftext[top,   at=\pgfpointanchor{#1}{south},y=-1mm]{#2}}%
\newcommand{\textleftof} [2]{\pgftext[right, at=\pgfpointanchor{#1}{west}, x=-1mm]{#2}}%

\makeatletter
\newcounter{r}
\newcommand{\tikzgrid}{%
  \pgfsetxvec{\pgfpoint{\tikz@node@distance}{0mm}}%
  \pgfsetyvec{\pgfpoint{0mm}{\tikz@node@distance}}%
  \tikz@matrix%
}
\newcommand{\tikz@matrix}[1]{\tikz@@matrix#1@}%
\def\tikz@@matrix#1@{\do@rows#1\\@\\}%
\def\do@rows#1\\{%
  \ifx#1@%
  \else%
    \setcounter{r}{0}%
    \do@columns#1&@&%
    \pgftransformshift{\pgfpointxy{-\ther}{-1}}%
    \expandafter\do@rows%
  \fi}%
\def\do@columns#1&{%
  \if#1@%
  \else%
    \stepcounter{r}%
    \pgftransformshift{\pgfpointxy{1}{0}}%
    #1;%
    \expandafter\do@columns%
  \fi}%
\makeatother

%%%%%%%%%%%%%%%%
% ENVIRONMENTS %
%%%%%%%%%%%%%%%%
\newif\ifnolabel
\newif\ifnocolor
\pgfkeys{
	/tikz/nomoccolor/.is if=nocolor,
	/tikz/nomoclabel/.is if=nolabel,
	/tikz/type style/.store in = \typeStyle,
	/tikz/type style = \scriptsize\textit,
	/tikz/label style/.store in = \labelStyle,
	/tikz/label style = \textbf,
	/tikz/function style/.store in = \funcStyle,
	/tikz/function style = \scriptsize,
	/tikz/moc/.store in = \MoC,
	/tikz/moc = none,
	/tikz/mocin/.store in = \MoCin,
	/tikz/mocin = none,
	/tikz/mocout/.store in = \MoCout,
	/tikz/mocout = none,
}
\pgfkeys{
	/applicative/.is family, /applicative,
	default/.style = {moc=none, reverse = false, type= , ni=1, no=1, nf=0, f1=$ f_1 $, f2=$ f_2 $, f3=$ f_3 $, f4=$ f_4 $, line sep=0pt},
 	moc/.estore in = \pMoc,
 	type/.estore in = \pType,
	ni/.estore in = \pNIn,
	no/.estore in = \pNOut,
	nf/.estore in = \pNFunc,
	f1/.estore in = \pFuncA,
	f2/.estore in = \pFuncB,
	f3/.estore in = \pFuncC,
	f4/.estore in = \pFuncD,
	line sep/.estore in = \pLineSep,
	reverse/.is toggle,
}
\pgfkeys{
	/primitive/.is family, /primitive,
	default/.style = {moc=none, reverse = false, type= , ni=1, no=1,},
 	moc/.estore in = \pMoc,
 	type/.estore in = \pType,
	ni/.estore in = \pNIn,
	no/.estore in = \pNOut,
	reverse shape/.is toggle,
	reverse/.is toggle,
}
\pgfkeys{
	/interface/.is family, /interface,
	default/.style = {mocin = none, mocout = none, },
 	mocin/.estore in = \pMocIn,
 	mocout/.estore in = \pMocOut,
}
\newcommand{\getmoclabel}[1]{
	\ifthenelse{\equal{#1}{sy}}{SY}{
	\ifthenelse{\equal{#1}{de}}{DE}{
	\ifthenelse{\equal{#1}{ct}}{CT}{
	\ifthenelse{\equal{#1}{sdf}}{SDF}{}}}} 
}

%%%%%%%%%%%%%%%%%
% GENERIC NODES %
%%%%%%%%%%%%%%%%%

% Generic applicative leaf process
% #1 = environment keys
% #2 = node name
% #3 = node position
% #4 = node label
\newcommand{\applicative}[4][]{
	\pgfkeys{/applicative, default, #1}%
	\node[inner sep=0pt] (#2_label) at (#3) {\labelStyle{#4}};
	\node[func\pNFunc, font=\funcStyle, yshift=2\sepq+\pLineSep, anchor=south] (f#2) at (#2_label.north) {
		\ifnum \pNFunc>0 \nodepart{fa} \pFuncA\else\fi
		\ifnum \pNFunc>1 \nodepart{fb} \pFuncB\else\fi
		\ifnum \pNFunc>2 \nodepart{fc} \pFuncC\else\fi
		\ifnum \pNFunc>3 \nodepart{fd} \pFuncD\else\fi
	};
	\node[anchor=north, yshift=-\pLineSep] (#2_type) at (#2_label.south) {\typeStyle{\pType\ifnolabel\else\getmoclabel{\pMoc}\fi}};
	\iftoggle{/applicative/reverse}
		{\node[i\pNIn o\pNOut, rotate=180, inner sep=0pt, fit=(f#2)(#2_label)(#2_type),] (#2) {};}
		{\node[i\pNIn o\pNOut, inner sep=0pt, fit=(f#2)(#2_label)(#2_type),] (#2) {};}
	\begin{pgfonlayer}{background}
		\node[applshape, draw, moc=\pMoc, inner sep=0pt, fit=(f#2)(#2_label)(#2_type)] {};
	\end{pgfonlayer}
}


% Generic primite box-shaped leaf process
% #1 = environment keys
% #2 = node name
% #3 = node position
% #4 = node label
\newcommand{\primitivebox}[4][]{
	\applicative[#1, nf=0, line sep=-2pt]{#2}{#3}{#4}
}
% Generic primite special-shaped leaf process
% #1 = environment keys
% #2 = node name
% #3 = node position
\newcommand{\primitivespecial}[4][]{
	\pgfkeys{/primitive, default, #1}%
	\pgfmathsetlength{\foo}{max(\pNIn,\pNOut)*5pt}
	\iftoggle{/primitive/reverse shape}
		{\node[\pType, draw, moc=\pMoc, inner sep=\foo, rotate=180,] (#2_shape) at (#3) {};}	
		{\node[\pType, draw, moc=\pMoc, inner sep=\foo,] (#2_shape) at (#3) {};}	
	\iftoggle{/primitive/reverse}
		{\node[i\pNIn o\pNOut, rotate=180, inner sep=0pt, fit=(#2_shape),] (#2) {};}
		{\node[i\pNIn o\pNOut, inner sep=0pt, fit=(#2_shape),] (#2) {};}
}


\newpage
% A library with graphical primitives for SDF process networks
%
% Author: George Ungureanu, KTH - Royal Institute of Technology, Sweden
% Version: 0.1
% Date: 2016/01/12
\NeedsTeXFormat{LaTeX2e}
\RequirePackage{mathtools}
\RequirePackage{amsmath}
\RequirePackage{amssymb}
\RequirePackage{stmaryrd}
\RequirePackage{graphicx}

\ProvidesPackage{forsyde-math}
              [2016/01/12 v0.1 SDF TikZ Library]

\newcommand\mdoubleplus{\ensuremath{\mathbin{+\mkern-10mu+}}}

\newcommand\vect[1]{\langle{#1}\rangle}
\newcommand\pc[1]{\mathtt{#1}}

\newcommand{\fcomb}{\oplus}
\newcommand{\fdelay}{\mathbin{\Delta}}

\newcommand{\fmap}{\mathbin{\rotatebox[origin=c]{45}{$\boxtimes$}}}
\newcommand{\fpipe}{\mathbin{\rotatebox[origin=c]{45}{$\boxbslash$}}}
\newcommand{\fred}{\mathbin{\rotatebox[origin=c]{45}{$\boxdot$}}}
\newcommand{\fscan}{\mathbin{\rotatebox[origin=c]{45}{$\boxast$}}}
\newcommand\fcat{\mdoubleplus}
\newcommand\fsel{\mathrel{@}}

% \newpage
% %% forsyde-plot.sty
%% Copyright 2016-2018 George Ungureanu
%
% This work may be distributed and/or modified under the
% conditions of the LaTeX Project Public License, either version 1.3
% of this license or (at your option) any later version.
% The latest version of this license is in
%   http://www.latex-project.org/lppl.txt
% and version 1.3 or later is part of all distributions of LaTeX
% version 2005/12/01 or later.
%
% This work has the LPPL maintenance status `maintained'.
% 
% The Current Maintainer of this work is George Ungureanu.
%
% This work consists of the files listed in LICENSE.

% A library with graphical primitives plotting ForSyDe signal 
% graphs
%
% Author: George Ungureanu, KTH - Royal Institute of Technology, Sweden
% Version: 0.1
% Date: 2016/08/13
\NeedsTeXFormat{LaTeX2e}
\RequirePackage{tikz}
\RequirePackage{pgfplots}
\RequirePackage{xparse}
\RequirePackage{calc}
\RequirePackage{etoolbox}
\usetikzlibrary{fit,calc,matrix}
\RequirePackage{environ}

\ProvidesPackage{forsyde-plot} [2017/05/20 v0.2 Signal Plot Library]
\usetikzlibrary{forsyde.shapes, forsyde.utils}


%%%%%%%%
% KEYS %
%%%%%%%%

\newcounter{signum}
\newif\ifgrid
\newif\iftimestamp
\newif\ifoverlap
\pgfkeys{/signal plot keys/.is family, /signal plot keys,
  show grid/.is if=grid,
  show timestamps/.is if=timestamp,
  overlap/.is if=overlap,
  draw inputs/.is if=drawinputs,
  draw outputs/.is if=drawoutputs,
  name/.estore in       = \plotName,
  label pos/.estore in  = \labelPos,
  step/.estore in       = \plotGridSize,
  xscale/.estore in     = \plotXScale,
  yscale/.estore in     = \plotYScale,
  xshift/.estore in     = \plotXShift,
  yshift/.estore in     = \plotYShift,
  signal sep/.estore in = \plotSep,
  at/.estore in         = \plotAt,
  anchor/.estore in     = \plotAnchor,
  grid/.style           ={show grid, step = {#1}, },
  timestamps/.style     ={show timestamps, step = {#1}, },
  grid and time/.style  ={show grid, show timestamps, step={#1}},
  left of/.style        ={at={#1}, anchor=east, xshift=-.5cm},
  right of/.style       ={at={#1}, anchor=west, xshift=.5cm},
  above of/.style       ={at={#1}, anchor=south},
  below of/.style       ={at={#1}, anchor=north},
  inputs/.style         ={left of={#1}},
  outputs/.style        ={right of={#1}},
  default/.style = {
    name=sigplot,
    label pos=mid, 
    step=5,
    xscale=.5,
    yscale=.5,
    xshift=0pt,
    yshift=0pt,
    signal sep=1,
    at={0,0},
    anchor=mid,
  }
}


%%%%%%%%%%%%%%%%%%%%%%%
% SYNCHRONOUS SIGNALS %
%%%%%%%%%%%%%%%%%%%%%%%
\NewEnviron{signalsSY}[2][]{%
  \pgfkeys{/signal plot keys, default, #1}%
  \matrix (\plotName) at (\plotAt) [%
    matrix of nodes, row sep=3pt, column sep=3pt,
    ampersand replacement=\&, nodes={align=right},
    xshift=\plotXShift, yshift=\plotYShift, anchor=\plotAnchor,
    ] {
      \BODY
    };
}

\NewDocumentCommand{\signalSY}{O{} >{ \SplitList {,} } m}{%
\ProcessList { #2 } {\syDrawEvents} \\
}

\NewDocumentCommand{\syDrawEvents}{ > {\SplitArgument{1}{:}} m }{%
\syEvent #1%
} 

\NewDocumentCommand{\syEvent} {m m} {#1 \&}

%% \inputSY[*][node keys] [<input port>] {events};
\NewDocumentCommand\inputSY{s O{} D<>{0,0} m}{%
  \node[anchor=south east, xshift=-.5cm, #2] (insig) at (#3) {#4};%
  \IfBooleanTF#1{\path[s,-|-=.9,->] (insig.south west) edge (#3);}{}%
}

% \outputSY[*][node keys] [<input port>] {events};
\NewDocumentCommand\outputSY{s O{} D<>{0,0} m}{%
  \node[anchor=south west, xshift=.5cm, #2] (outsig) at (#3) {#4};%
  \IfBooleanTF#1{\path[s,-|-=.9,<-] (outsig.south east) edge (#3);}{}%
}
 
%%%%%%%%%%%%%%%%%%%%%%%%%%
% DISCRETE EVENT SIGNALS %
%%%%%%%%%%%%%%%%%%%%%%%%%%
  
\NewDocumentEnvironment{signalsDE}{O{} m}{%
  \pgfkeys{/signal plot keys, default, #1}%
  \setcounter{signum}{0}
  \def\lastTag{#2}
  \node[%
    xshift=\plotXShift, yshift=\plotYShift,
    inner sep=1.5pt, anchor=\plotAnchor,
  ] (sigdrawing) at (\plotAt) %
  \bgroup
    \begin{tikzpicture}[%
      draw=black,xscale=\plotXScale, yscale=\plotYScale] %
      \tikzstyle{time}=[coordinate]%
    }{%
      \ifgrid%
      \pgfmathsetmacro\ymin{-\plotSep*\thesignum + 0.5}
      \foreach \x [count=\i] in {0,\plotGridSize,...,\lastTag} {
        \draw[black!60, dashed,line width=0.2pt]
        (\x,0.5) -- (\x,\ymin);
      }
      \else\fi%
    \end{tikzpicture}%
  \egroup;
  \node[ports e\thesignum w\thesignum, inner sep=0pt,
    fit=(sigdrawing)] (\plotName) {};
  \iftimestamp
  \node[anchor=south west, inner sep=0] at (\plotName .north west) {%
    \begin{tikzpicture}[%
      draw=black, %
      xscale=\plotXScale, yscale=\plotYScale,
      ] %
      \foreach \x [count=\i] in {0,\plotGridSize,...,\lastTag} {
        \pgfmathsetmacro\timestamp{(\i-1)*\plotGridSize}
        \node[anchor=south] at (\x,0.5) {%
          \textbf{\scriptsize\timestamp}};
      }
    \end{tikzpicture}
  };
  \else\fi
  % \ifdrawinputs
  % \foreach \i in {1,...,\thesignum} {
  %   \path (sigplot.e\i) edge[s,-|-,->] (\plotAt.w\i);
  % }
  % \else\fi
  % \ifdrawoutputs
  % \foreach \i in {1,...,\thesignum} {
  %   \path (sigplot.w\i) edge[s,-|-,<-] (\plotAt.e\i);
  % }
  % \else\fi
}

\usepackage{catchfile}

\newif\iftruncatevalue
\newif\iflastlabel
\pgfkeys{/designalkeys/.is family, /designalkeys,
  trunc/.is if=truncatevalue,
  last label/.is if=lastlabel,
  name/.estore in = \deName,
  default/.style = {
    trunc=false,
    last label=true,
    name=,
  }
}

\ExplSyntaxOn
\NewDocumentCommand{\signalDE}{s O{} m }{%
  \pgfkeys{/designalkeys, default, #2}%
  %%% prepare drawing of signal %%%
  \pgfmathsetmacro{\deYPos}{-\thesignum * \plotSep}
  \path (0,\deYPos) node[left] {\scriptsize\deName} node[time] (t_cur) {};
  %%% test if reading from file or string %%%
  \IfBooleanTF{#1}
  { \IfFileExists{#3}
    {\CatchFileDef{\deTokensString}{#3}{\endlinechar=-1 }}
    {\PackageError{forsyde-plot}
      {File '#3' does not exist!}
      {Please provide a valid file path to the 'signalDE' command}}
    %%% start drawing. expand the argument (once) %%%
    \desig_toklist:o { \deTokensString }
  }
  { %%% start drawing %%%
    \desig_toklist:n { #3 }
  }
  \firstitemtrue%
  \stepcounter{signum}
}

\seq_new:N \l_desig_toklist_input_seq

\cs_new_protected:Npn \desig_toklist:n #1
{
  \seq_set_split:Nnn \l_desig_toklist_input_seq {,} { #1 }
  \seq_map_inline:Nn \l_desig_toklist_input_seq
  {
    \deDrawEvents{##1}
  }
}
\cs_generate_variant:Nn \desig_toklist:n { o }
\ExplSyntaxOff

\newif\iffirstitem
\firstitemtrue   

\NewDocumentCommand{\deDrawEvents}{ > {\SplitArgument{1}{:}} m }{%
  \iffirstitem\firstitemfalse\deFirstEvent #1
  \else\deEvent #1
  \fi
} 

\NewDocumentCommand{\deFirstEvent}{m m} {%
  \def\dePrevValue{#1}%
}

\NewDocumentCommand{\deEvent}{m m} {%
  \iftruncatevalue\pgfmathtruncatemacro\valLabel{\dePrevValue}
  \else\def\valLabel{\dePrevValue}\fi
  \ifdim#2 pt<\lastTag pt%
    \draw (t_cur) -- ++(.05, .3) -- (#2-.05,.3+\deYPos) -- ++(.05,-.3)
      -- ++(-.05,-.3) -- ($(t_cur)+(.05,-.3)$) -- cycle; %
    \path (t_cur) -- node[anchor=\labelPos] {\scriptsize\valLabel}
      (#2,\deYPos) node[time] (t_cur) {};
    \def\dePrevValue{#1}
  \else%
    \iflastlabel\else\def\valLabel{}\fi
    \draw (t_cur) -- ++(.05, .3) -- (\lastTag,.3+\deYPos);
    \draw (t_cur) -- ++(.05, -.3) -- (\lastTag,-.3+\deYPos);
    \path (t_cur) -- node[anchor=\labelPos] {\scriptsize\valLabel}
      (\lastTag,\deYPos) node[time] (signal-\thesignum) {};
  \fi
}


%%%%%%%%%%%%%%%%%%%%%%%%%%%
% CONTINUOUS TIME SIGNALS %
%%%%%%%%%%%%%%%%%%%%%%%%%%%

\newcommand{\Colors}{{%
"E41A1C",%
"377EB8",%
"4DAF4A",%
"984EA3",%
"FF7F00",%
"FFFF33",%
"A65628",%
"F781BF"
}}

\def\ctPlotWidth{4}
%% \inputDE[*][node keys] [<position>] {span};
\NewDocumentEnvironment{signalsCT}{O{} O{0} m}{%
  \pgfkeys{/signal plot keys, default, #1}%
  \setcounter{signum}{0}
  \pgfmathsetmacro{\firstTag}{#2}
  \pgfmathsetmacro{\lastTag}{#3}
  \node[%
    xshift=\plotXShift, yshift=\plotYShift,
    inner sep=1.5pt, anchor=\plotAnchor,
  ] (sigdrawing) at (\plotAt) \bgroup
  \begin{tikzpicture}[%
    draw=black,xscale=\plotXScale, yscale=\plotYScale] %
    \tikzstyle{time}=[coordinate]%
  }{%
    \ifgrid
    \pgfmathsetmacro{\rescaleX}{\ctPlotWidth /(\lastTag -\firstTag)}
    \pgfmathsetmacro{\ymin}{-\plotSep*(\thesignum - 1) * 1.5 - .5}
    \pgfmathsetmacro{\gridstep}{\rescaleX * \plotGridSize}
    \foreach \x in {0,\gridstep,...,\ctPlotWidth} {
      \draw[black!60,dashed,line width=0.2pt] (\x,0.5) -- (\x,\ymin);
    }
    \else\fi%
  \end{tikzpicture}%
  \egroup;
  \node[ports e\thesignum w\thesignum,inner sep=0pt,
    fit=(sigdrawing)] (\plotName) {};
  \iftimestamp
  \pgfmathsetmacro{\rescaleX}{\ctPlotWidth /(\lastTag -\firstTag)}
  \pgfmathsetmacro{\gridstep}{\rescaleX * \plotGridSize}
  \node[anchor=south east, inner sep=0] at (\plotName .north east) {%
    \begin{tikzpicture}[draw=black,%
      xscale=\plotXScale, yscale=\plotYScale] 
      \foreach \x [count=\i] in {0,\gridstep,...,\ctPlotWidth} {
        \pgfmathsetmacro\timestamp{(\i -1) *\plotGridSize +\firstTag}
        \node[anchor=south,gray] at (\x,0.5) {%
          \tiny\pgfmathprintnumber{\timestamp}};
      }
      \node[anchor=south,gray] at (\ctPlotWidth,0.5) {%
        \tiny\phantom{\pgfmathprintnumber{\lastTag}}};
    \end{tikzpicture}
  };
  \else\fi
  % \ifdrawinputs
  % \foreach \i in {1,...,\thesignum} {
  %   \path (\plotName.e\i) edge[s,-|-,->] (\plotAt.w\i);
  % }
  % \else\fi
  % \ifdrawoutputs
  % \foreach \i in {1,...,\thesignum} {
  %   \path (\plotName.w\i) edge[s,-|-,<-] (\plotAt.e\i);
  % }
  % \else\fi
}


\newif\ifdrawordinate
\newif\ifdrawoutline
\pgfkeys{/ctsignalkeys/.is family, /ctsignalkeys,
  draw ordinate/.is if=drawordinate,
  outline/.is if=drawoutline,
  % step/.estore in         = \plotSamp,
  ymax/.estore in         = \plotYMax,
  ymin/.estore in         = \plotYMin,
  name/.estore in         = \ctName,
  ordinate pos/.estore in = \plotOrdPos,
  ordinate/.style  ={draw ordinate, ordinate pos={#1}},
  line style/.estore in   = \lineStyle,
  default/.style = {
    outline=false,
    % step=.1,
    ymin=0,
    ymax=1,
    name=,
    ordinate pos=0,
    line style=
  }
}

\ExplSyntaxOn
\NewDocumentCommand{\signalCT}{s O{} m}{%
  \pgfkeys{/ctsignalkeys, default, #2}%
  % determine the Y position
  \ifoverlap\pgfmathsetmacro{\deYPos}{0}
  \else\pgfmathsetmacro{\deYPos}{-\thesignum * \plotSep * 1.5}\fi
  % determine the scale
  % \pgfmathsetmacro{\plotYMin}{#2}\pgfmathsetmacro{\plotYMax}{#3}
  \pgfmathsetmacro{\rescaleY}{1/(\plotYMax - \plotYMin)}
  \pgfmathsetmacro{\rescaleX}{\ctPlotWidth /(\lastTag -\firstTag)}
  \pgfmathsetmacro{\plotShift}{(\plotYMax + \plotYMin)/2}
  % \pgfmathsetmacro{\plotStep}{\ctPlotWidth / \lastTag * \plotSamp}
  % determine the color
  \pgfmathsetmacro{\thecurrentcolor}{\Colors[\value{signum}]}
  \definecolor{currentcolor}{HTML}{\thecurrentcolor}
  % start plotting
  \path (0,\deYPos) node[left] {} node[time] (t_start) {};
  \IfBooleanTF{#1}
  { \IfFileExists{#3}
    {\CatchFileDef{\ctTokensString}{#3}{\endlinechar=-1 }}
    {\PackageError{forsyde-plot}
      {File~'#3'~does~not~exist!}
      {Please provide a valid file path to the 'signalDE' command}}
    %%% start drawing. expand the argument (once) %%%
    \ctsig_toklist:o { \ctTokensString }
  }
  { %%% start drawing %%%
    \ctsig_toklist:n { #3 }
  }
  \postDrawPlot
  % update state
  \firstitemtrue%
  \stepcounter{signum}
}
\seq_new:N \l_ctsig_toklist_input_seq

\cs_new_protected:Npn \ctsig_toklist:n #1
{
  \seq_set_split:Nnn \l_ctsig_toklist_input_seq {,} { #1 }
  \seq_map_inline:Nn \l_ctsig_toklist_input_seq
  {
    \ctDrawEvents{##1}
  }
}
\cs_generate_variant:Nn \ctsig_toklist:n { o }
\ExplSyntaxOff

\newcommand{\postDrawPlot}{
  \ifdrawordinate
    \pgfmathsetmacro{\ordYPos}{(\plotOrdPos - \plotShift) *\rescaleY +\deYPos}
    \draw[->,line width=.1pt]%
    (0,\ordYPos) node[left] (t_start) {\tiny\pgfmathprintnumber{\plotOrdPos}}
    -- (\ctPlotWidth,\ordYPos);
  \else\fi
  \ifdrawoutline
    \draw[ultra thin, gray] (0,\deYPos)
    --++(0,.5) node[left] {\tiny\pgfmathprintnumber{\plotYMax}}
    --++(\ctPlotWidth,0) ++(0,-1)
    --++(-\ctPlotWidth,0) node[left] {\tiny\pgfmathprintnumber{\plotYMin}}
    --++(0,.5);
  \else\fi
  \node[anchor=east] at (t_start.west) {\tiny\it\ctName};  
}

\NewDocumentCommand{\ctDrawEvents}{ > {\SplitArgument{1}{:}} m }{%
  \iffirstitem\ctFirstEvent #1
  \else\ctEvent #1
  \fi
} 

\NewDocumentCommand{\ctFirstEvent}{ m m }{%
  \ifdim#2 pt<\firstTag pt\else% 
  % \ifnum0#2<\firstTag\else
  \pgfmathsetmacro{\plotYPos}{(#1 -\plotShift)*\rescaleY +\deYPos}
  \node[time] (t_cur) at (0,\plotYPos){};
  \firstitemfalse
  \fi
}

\NewDocumentCommand{\ctEvent}{ m m }{%
  \ifdim#2 pt>\lastTag pt\else% 
  % \ifnum0#2>\lastTag\else
  \pgfmathsetmacro{\plotYPos}{(#1 -\plotShift)*\rescaleY +\deYPos}
  \pgfmathsetmacro{\plotXPos}{(#2 -\firstTag) *\rescaleX}
  \draw[color=currentcolor, thin, \lineStyle]
    (t_cur) -- (\plotXPos,\plotYPos) node[time] (t_cur) {};
  \fi
}

\newpage
\section{The \texttt{forsyde-legacy} package}
\label{sec:legacy-package}

This package offers an API for the legacy commands defined in older versions of the \ForSyDeLaTeX utilities. This way, documents compiled with old commands can be compiled with the newer versions of their respective library.

\subsection{\texttt{forsyde-tikz v0.3} or prior}
\label{sec:forsyde-tikz-v0.3}

Although from \texttt{v0.4} onward the draw commands have been heavily modified, the old commands could be mapped to the new API.

\begin{lstlisting}
\primitive[keys]         {id}{pos}{label}
\primitiven[keys]        {id}{pos}{label}
\leafstd[keys]           {id}{pos}{label}
\leafcustom[keys]        {id}{pos}
\compositestd[keys]      {id}{clustered nodes}{label}
\compositebbox[keys]     {id}{pos}{label}
\patterncluster[keys]    {id}{clustered nodes}{label}
\patternnodestd[keys]    {id}{pos}
\patternnodecustom[keys] {id}{pos}
\end{lstlisting}

\subsection{\texttt{forsyde-pc v0.3} or prior}
\label{sec:forsyde-pc-v0.3}

This package is obsolete and used to hold helpers associated to some \ForSyDe process constructors.

\begin{lstlisting}
\delay    [moc=,f1=,inner sep=,reverse]        {id}{pos}{label}
\delayn   [moc=,f1=,f2=,inner sep=,reverse]    {id}{pos}{label}
\map      [moc=,f1=,inner sep=,reverse]        {id}{pos}{label}
\comb     [moc=,f1=,inner sep=,reverse]        {id}{pos}{label}
\combII   [moc=,f1=,inner sep=,reverse]        {id}{pos}{label}
\combIII  [moc=,f1=,inner sep=,reverse]        {id}{pos}{label}
\combIV   [moc=,f1=,inner sep=,reverse]        {id}{pos}{label}
\scanl    [moc=,f1=,f2=,inner sep=,reverse]    {id}{pos}{label}
\scanlII  [moc=,f1=,f2=,inner sep=,reverse]    {id}{pos}{label}
\scanlIII [moc=,f1=,f2=,inner sep=,reverse]    {id}{pos}{label}
\scanld   [moc=,f1=,f2=,f3=,inner sep=,reverse]{id}{pos}{label}
\scanldII [moc=,f1=,f2=,f3=,inner sep=,reverse]{id}{pos}{label}
\scanldIII[moc=,f1=,f2=,f3=,inner sep=,reverse]{id}{pos}{label}
\moore    [moc=,f1=,f2=,f3=,inner sep=,reverse]{id}{pos}{label}
\mooreII  [moc=,f1=,f2=,f3=,inner sep=,reverse]{id}{pos}{label}
\mooreIII [moc=,f1=,f2=,f3=,inner sep=,reverse]{id}{pos}{label}
\mealy    [moc=,f1=,f2=,f3=,inner sep=,reverse]{id}{pos}{label}
\mealyII  [moc=,f1=,f2=,f3=,inner sep=,reverse]{id}{pos}{label}
\mealyIII [moc=,f1=,f2=,f3=,inner sep=,reverse]{id}{pos}{label}
\source   [moc=,f1=,f2=,inner sep=,reverse]    {id}{pos}{label}
\filter   [moc=,f1=,f2=,inner sep=,reverse]    {id}{pos}{label}
\hold     [moc=,f1=,inner sep=,reverse]        {id}{pos}{label}
\fillS    [moc=,f1=,f2=,inner sep=,reverse]    {id}{pos}{label}

\zip     [moc=,reverse]{id}{pos}
\zipIII  [moc=,reverse]{id}{pos}
\zipIV   [moc=,reverse]{id}{pos}
\zipV    [moc=,reverse]{id}{pos}
\zipVI   [moc=,reverse]{id}{pos}
\unzip   [moc=,reverse]{id}{pos}
\unzipIII[moc=,reverse]{id}{pos}
\unzipIV [moc=,reverse]{id}{pos}
\unzipV  [moc=,reverse]{id}{pos}
\unzipVI [moc=,reverse]{id}{pos}

\domaininterface[moc=,reverse]          {id}{pos}
\mocinterface   [mocin=,mocout=,reverse]{id}{pos}

\composite[ni=,no=,inner xsep=,inner ysep=,reverse] {id}{included}{label}
\blackbox [ni=,no=,inner xsep=,inner ysep=,reverse] {id}{included}{label}

\farm     [ni=,no=,inner xsep=,inner ysep=,reverse]                {id}{included}{label}
\farmI    [ni=,no=,f1=,inner xsep=,inner ysep=,reverse]            {id}{included}{label}
\farmII   [ni=,no=,f1=,f2=,inner xsep=,inner ysep=,reverse]        {id}{included}{label}
\farmIII  [ni=,no=,f1=,f2=,f3=,inner xsep=,inner ysep=,reverse]    {id}{included}{label}
\farmIV   [ni=,no=,f1=,f2=,f3=,f4=,inner xsep=,inner ysep=,reverse]{id}{included}{label}
\pipe     [ni=,no=,inner xsep=,inner ysep=,reverse]                {id}{included}{label}
\pipeI    [ni=,no=,f1=,inner xsep=,inner ysep=,reverse]            {id}{included}{label}
\pipeII   [ni=,no=,f1=,f2=,inner xsep=,inner ysep=,reverse]        {id}{included}{label}
\pipeIII  [ni=,no=,f1=,f2=,f3=,inner xsep=,inner ysep=,reverse]    {id}{included}{label}
\pipeIV   [ni=,no=,f1=,f2=,f3=,f4=,inner xsep=,inner ysep=,reverse]{id}{included}{label}
\reduce   [ni=,no=,inner xsep=,inner ysep=,reverse]                {id}{included}{label}
\reduceI  [ni=,no=,f1=,inner xsep=,inner ysep=,reverse]            {id}{included}{label}
\reduceII [ni=,no=,f1=,f2=,inner xsep=,inner ysep=,reverse]        {id}{included}{label}
\reduceIII[ni=,no=,f1=,f2=,f3=,inner xsep=,inner ysep=,reverse]    {id}{included}{label}
\reduceIV [ni=,no=,f1=,f2=,f3=,f4=,inner xsep=,inner ysep=,reverse]{id}{included}{label}

\unzipx     [reverse]         {id}{position}
\zipx       [reverse]         {id}{position}
\unzipv     [reverse]         {id}{position}
\zipv       [reverse]         {id}{position}
\splitatv   [f1=,reverse]     {id}{position}
\catv       [reverse]         {id}{position}
\oddsv      [reverse]         {id}{position}
\evensv     [reverse]         {id}{position}
\reversev   [reverse]         {id}{position}
\groupv     [reverse]         {id}{position}
\concatv    [reverse]         {id}{position}
\filteridxv [f1=,reverse]     {id}{position}
\gatherv    [f1=,f2=,reverse] {id}{position}
\gatherAdpv [f1=,f2=,reverse] {id}{position}
\selectv    [reverse]         {id}{position}
\distributev[f1=,reverse]     {id}{position}
\filterv    [f1=,reverse]     {id}{position}
\getv       [f1=,reverse]     {id}{position}

\visualoddsv   [reverse]{id}{pos}
\visualevensv  [reverse]{id}{pos}
\visualreversev[reverse]{id}{pos}
\visualgroupv  [reverse]{id}{pos}
\visualconcatv [reverse]{id}{pos}
\end{lstlisting}


%%% Local Variables:
%%% TeX-command-default: "Make"
%%% mode: latex
%%% TeX-master: "../refman"
%%% End:



\end{document}
%%% Local Variables:
%%% TeX-command-default: "Make"
%%% mode: latex
%%% TeX-master: t
%%% End:
