%% tikzlibraryforsyde.paths.code.tex
%% Copyright 2016-2018 George Ungureanu
%
% This work may be distributed and/or modified under the
% conditions of the LaTeX Project Public License, either version 1.3
% of this license or (at your option) any later version.
% The latest version of this license is in
%   http://www.latex-project.org/lppl.txt
% and version 1.3 or later is part of all distributions of LaTeX
% version 2005/12/01 or later.
%
% This work has the LPPL maintenance status `maintained'.
% 
% The Current Maintainer of this work is George Ungureanu.
%
% This work consists of the files listed in LICENSE.

% pgf/tikz library for ForSyDe signals
%
% Author: George Ungureanu, KTH - Royal Institute of Technology, Sweden
% Version: 0.4
% Date: 2017/01/28

\usetikzlibrary{forsyde.utils}

\newlength{\xTemp}
\newlength{\inTemp}
\newlength{\lTemp}
\newlength{\startFunc}

% Global keys used in this library
\pgfkeys{
  /tikz/token pos/.store in = \tokPos,
  /tikz/token pos = 0.5,
  /tikz/deviate/.store in = \pDeviate,
  /tikz/deviate = 0pt,
  /tikz/alwaystrue/.is toggle,
  /tikz/alwaystrue = true,
  /tikz/as/.store in = \pIntName,
  /tikz/as = int,
}

% Parser for decorating paths with token representation. It parses
% strings "scalar-vector-(vector)" and draws the corresponding
% symbols.
\ExplSyntaxOn

% Main parser function. First counts the tokens then draws them.
% #1 = delimiter character, #2 = string
\NewDocumentCommand{\tokens}{m} {%
  \youra_count_char:nn { - } { #1 }
  \drawtokens{ #1 }
}
\int_new:N \l_youra_count_char_int
\cs_new_protected:Npn \youra_count_char:nn #1 #2 {%
  \regex_count:nnN { #1 } { #2 } \l_youra_count_char_int
}
% Splits the string into individual tokens and draws them
\NewDocumentCommand{\drawtokens}{ >{ \SplitList { - } } m } {%
 \pgfmathsetlength{\xTemp}{-\l_youra_count_char_int * \tokensize}
 \pgfmathsetlength{\lTemp}{\tokensize + 0.3pt}
 \setlength{\inTemp}{\halftokensize}
 \ProcessList { #1 } { \davs__tokens_drawbox_rec:n }
 \pgfmathsetlength{\xTemp}{-\l_youra_count_char_int * \tokensize}
 \setlength{\inTemp}{\halftokensize}
 \ProcessList { #1 } { \davs__tokens_drawtok_rec:n }
}
% Checks whether the token is wrapped as a function and draws boxes
% around the respective tokens.
\cs_new_protected:Nn \davs__tokens_drawbox_rec:n {%
  \IfSubStr { #1 } { ( } % beginning of functions
    { \pgfmathsetlength{\startFunc}{ \xTemp - 0.3pt - \tokensize  } }
    {  }
  \IfSubStr { #1 } { ) } % end of function
    { \pgfmathsetlength{\inTemp}{ \xTemp + 0.3pt + \tokensize }
      \pgfsetlinewidth{0.2pt}
      \draw[ fill=\defaultfillcolor] (\startFunc,-\lTemp) rectangle (\inTemp,\lTemp); }
    {  }
  \pgfmathsetlength{\xTemp}{\xTemp+ 2 * \tokensize}
}
% Draws a symbol for each token
\cs_new_protected:Nn \davs__tokens_drawtok_rec:n {%
  \drawTokenSymbol{#1} % the actual drawing function as case switch
  \pgfmathsetlength{\xTemp}{\xTemp+ 2 * \tokensize}
}
\ExplSyntaxOff

% Case switch for drawing symbols
\makeswitch[]\drawTokenSymbol
\addcase\drawTokenSymbol{scalar}{\node[scalartokenshape, draw] at (\xTemp,0) {};}
\addcase\drawTokenSymbol{vector}{\node[vectortokenshape, draw] at (\xTemp,0) {};}
\addcase\drawTokenSymbol{function}{\node[functiontokenshape, draw] at (\xTemp,0){};}


% Collection for tikz draw functions for paths.
\tikzset{%
  % simple switch for color, but as a pgf .code key
  getsignalmoc/.code={%
    \ifnocolor \defaultdrawcolor \else
    \ifstrequal{#1}{sy}{\tikzset{sycolor}}{%
      \ifstrequal{#1}{de}{\tikzset{decolor}}{%
        \ifstrequal{#1}{ct}{\tikzset{ctcolor}}{%
          \ifstrequal{#1}{sdf}{\tikzset{sdfcolor}}{}}}}
    \fi
  },%
  % port shape
  port/.style    = {fill=\defaultdrawcolor,inner sep=2pt},
  % path styles: s=signal, sn=n-arry signal,
  s/.style args  = {#1}{getsignalmoc={#1}, line width=\signalpathlinewidth, >= stealth},
  sn/.style args = {#1}{getsignalmoc={#1}, line width=.5,  >= stealth, double distance=1.3pt},
  % path styles: v=vector, vn=n-arry vector,
  v/.style args  = {#1}{getsignalmoc={#1}, line width=\vectorpathlinewidth, >= stealth},
  vn/.style args = {#1}{getsignalmoc={#1}, line width=2, double distance=2pt, >= stealth, double},
  % path styles: f=function, fn=n-arry function
  f/.style       = {line width=\functionpathlinewidth, >= open triangle 90, densely dotted},
  fn/.style      = {line width=.4, >= open triangle 90, densely dotted, double distance=1pt, double},
  % source port node (as decoration)
  srcport/.style={%
    shorten <=2pt,%
    decoration={%
      name=markings,
      mark=at position 0 with {\node[port]{};}%
    },%
    postaction=decorate,
  },
  % destination port node (as decoration)
  dstport/.style={%
    shorten >=2pt,%
    decoration={%
      name=markings,
      mark=at position 1 with {\node[port]{};}%
    },%
    postaction=decorate,
  },
  % token funcion. the token parser with its argument
  token/.style={%
    decoration={%
      name=markings,
      mark=at position \tokPos with {\tokens{#1}}%
    },
    postaction=decorate,
  },
  % Intersection between this path and a named line, named with the
  % "as" option (see fkeys). If no "as" mentioned, the default name
  % currently is "int".
  intersect/.style = {%
    decoration={show path construction, lineto code={
      \path[name path=this path] (\tikzinputsegmentfirst) -- (\tikzinputsegmentlast); 
      \path[name intersections={of=this path and #1,by={\pIntName}}];%
      }
    },
    decorate,
  },%
  % Helper function for drawing a transition from one path style to
  % another base on the intersection with a line. Basically saves the
  % user from naming the intersection and to explicitly draw the paths
  % herself.
  trans/.style n args={3}{%
    decoration={show path construction, lineto code={
      \path[draw opacity=0, name path=this path] (\tikzinputsegmentfirst) -- (\tikzinputsegmentlast); 
      \path[name intersections={of=this path and #2,by={int-1}}];%
      \path (\tikzinputsegmentfirst) edge[#1] (int-1)
            (int-1) edge[#3] (\tikzinputsegmentlast); 
      }
    },
    decorate,
  },%
  % Helpers for drawing spline paths. Character '|' represent
  % vertical path (turn) and character '-' represents horisontal path
  % (turn). Some are affected by the "deviate" key (see code).
  -|/.style={ to path={ (\tikztostart) -| (\tikztotarget) \tikztonodes }},
  |-/.style={ to path={ (\tikztostart) |- (\tikztotarget) \tikztonodes }},
  -|-/.style={ to path={ (\tikztostart) 
	-| ([xshift=\pDeviate]$(\tikztostart)!#1!(\tikztotarget)$) 
	|- (\tikztotarget) 
	\tikztonodes }
  }, -|-/.default=0.5,
  |-|/.style={to path={ (\tikztostart) 
	|- ($(\tikztostart)!#1!(\tikztotarget)$) 
	-| (\tikztotarget)
    \tikztonodes }
  }, |-|/.default=0.5,
  -|-|/.style args = {#1}{ to path={ (\tikztostart) 
	-| ($(\tikztostart)!#1!(\tikztotarget)$) 
	|- ([yshift=\pDeviate]\tikztotarget) 
	-| (\tikztotarget) 
	\tikztonodes }
  }, -|-|/.default=0.5,
  -|-|-/.style args = {#1:#2}{ to path={ (\tikztostart) 
	-| ($(\tikztostart)!#1!(\tikztotarget)$) 
	|- ([yshift=\pDeviate]$(\tikztostart)!#2!(\tikztotarget)$) 
	-| ([yshift=\pDeviate]$(\tikztostart)!#2!(\tikztotarget)$)
	|- (\tikztotarget) 
	\tikztonodes }
  }, -|-|-/.default=0.3:0.7,
  |-|-|/.style args = {#1:#2}{ to path={ (\tikztostart) 
	|- ($(\tikztostart)!#1!(\tikztotarget)$) 
	-| ([xshift=\pDeviate]$(\tikztostart)!#2!(\tikztotarget)$) 
	|- ([xshift=\pDeviate]$(\tikztostart)!#2!(\tikztotarget)$)
	-| (\tikztotarget) 
	\tikztonodes }
  }, |-|-|/.default=0.3:0.7,
}

% Some edge drawing helpers (not very pretty, since the argument syntax is strict).
\def\signal[#1] (#2) #3 (#4);{
        \draw (#2) edge[#1, #3, s=\MoC,] (#4);
}
\def\signaln[#1] (#2) #3 (#4);{
        \draw (#2) edge[#1, #3, sn=\MoC,] (#4);
}
\def\vector[#1] (#2) #3 (#4);{
        \draw (#2) edge[#1, #3, v=\MoC,] (#4);
}
\def\vectorn[#1] (#2) #3 (#4);{
        \draw (#2) edge[#1, #3, vn=\MoC,] (#4);
}
\def\function[#1] (#2) #3 (#4);{
        \draw (#2) edge[#1, #3, f,] (#4);
}
\def\functionn[#1] (#2) #3 (#4);{
        \draw (#2) edge[#1, #3, fn,] (#4);
}

\endinput
