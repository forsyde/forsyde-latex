%% tikzlibraryforsyde.pictures.layered.code.tex
%% Copyright 2016-2018 George Ungureanu
%
% This work may be distributed and/or modified under the
% conditions of the LaTeX Project Public License, either version 1.3
% of this license or (at your option) any later version.
% The latest version of this license is in
%   http://www.latex-project.org/lppl.txt
% and version 1.3 or later is part of all distributions of LaTeX
% version 2005/12/01 or later.
%
% This work has the LPPL maintenance status `maintained'.
% 
% The Current Maintainer of this work is George Ungureanu.
%
% This work consists of the files listed in LICENSE.

\newlength{\currentradius}
\newlength{\layerArrowHalf}
\newlength{\layerArrowQuarter}


% \newcommand{\gettikzx}[2]{%
%   \tikz@scan@one@point\pgfutil@firstofone#1\relax
%   \edef#2{\the\pgf@x}%
% }
% % Get y-coordinate of node
% \newcommand{\gettikzy}[2]{%
%   \tikz@scan@one@point\pgfutil@firstofone#1\relax
%   \edef#2{\the\pgf@y}%
% }
% % Get x- and y- coordinate of node
% \newcommand{\gettikzxy}[3]{%
%   \tikz@scan@one@point\pgfutil@firstofone#1\relax
%   \edef#2{\the\pgf@x}%
%   \edef#3{\the\pgf@y}%
% }

\pgfkeys{
  /forsyde-atom layer keys/.is family, 
  /forsyde-atom layer keys/left indent/.store in  = \layerLeftIndent,
  /forsyde-atom layer keys/right indent/.store in = \layerRightIndent,
  /forsyde-atom layer keys/label shift/.store in  = \layerLabelShift,
  /forsyde-atom layer keys/layer radius/.store in = \layerRadius,
  /forsyde-atom layer keys/default/.style = {
    left indent=  3pt,
    right indent= 0pt,
    label shift= -4ex,
    layer radius= 2cm,
  },
  /forsyde-atom arrow keys/.is family, 
  /forsyde-atom arrow keys/width/.store in  = \layerArrowWidth,
  /forsyde-atom arrow keys/height/.store in = \layerArrowHeight,
  /forsyde-atom arrow keys/length/.store in = \layerArrowLength,
  /forsyde-atom arrow keys/default/.style = {
    width  = 3.8cm,
    height = 1.8cm,
    length = 1cm,
  }  
}
\newcommand\forsydeAtomSignalArrow[2][]{%
  \pgfkeys{/forsyde-atom arrow keys, default, #1}%
  \setlength{\layerArrowHalf}{\layerArrowHeight}
  \setlength{\layerArrowHalf}{.5\layerArrowHalf}
  \setlength{\layerArrowQuarter}{\layerArrowHeight}
  \setlength{\layerArrowQuarter}{.25\layerArrowQuarter}
  \draw[fill=white]
  ($(#2)+(-\layerArrowQuarter,\layerArrowHeight)-(0,\layerArrowQuarter)$) coordinate (arrow-base)
  --++(\layerArrowQuarter,\layerArrowQuarter)
  --++(0,-\layerArrowHeight)--++(-\layerArrowHeight,0)
  --++(\layerArrowQuarter,\layerArrowQuarter)
  --++(-\layerArrowLength,\layerArrowLength) coordinate (arrow-south east)
  --++(-\layerArrowWidth,0)  coordinate (arrow-south west)
  --++(0,\layerArrowHeight)  coordinate (arrow-north west)
  --++(\layerArrowWidth,0)   coordinate (arrow-north east)
  --cycle;
  \coordinate (arrow-south)  at ($(arrow-south west)!.5!(arrow-south east)$);
  \coordinate (arrow-north)  at ($(arrow-north west)!.5!(arrow-north east)$);
  \coordinate (arrow-east)   at ($(arrow-south west)!.5!(arrow-north west)$);
  \coordinate (arrow-west)   at ($(arrow-north east)!.5!(arrow-south east)$);
  \coordinate (arrow-center) at ($(arrow-south)!.5!(arrow-north)$);
}

% obligatory AFTER \forsydeAtomSignalArrow
\newcommand\forsydeAtomSignalVector[1]{
  \coordinate[xshift=2pt,yshift=2pt] (vector-north east1) at (arrow-north east);
  \coordinate[xshift=4pt,yshift=4pt] (vector-north east2) at (arrow-north east);
  \path [name path=vector1] (vector-north east1) -- ($(arrow-base)+(2pt,2pt)$);
  \path [name path=vector2] (vector-north east2) -- ($(arrow-base)+(4pt,4pt)$);
  \path [name intersections={of=vector1 and #1,by={vector-south east1}}];
  \path [name intersections={of=vector2 and #1,by={vector-south east2}}];
  \draw ($(arrow-north west)+(2pt,0)$) --++(0,2pt) -- (vector-north east1) -- (vector-south east1);
  \draw ($(arrow-north west)+(4pt,2pt)$) --++(0,2pt) -- (vector-north east2) -- (vector-south east2);
}

\NewDocumentCommand{\forsydeAtomMakeLayers}{O{} >{ \SplitList { , } } m} {%
  \pgfkeys{/forsyde-atom layer keys, default, #1}%
  \setlength{\currentradius}{\layerRadius}
  \setlength{\currentradius}{.7\currentradius+\currentradius}
  \pgfmathtruncatemacro{\currentLayer}{1}
  \ProcessList { #2 } { \forsydeAtomSetLayer }
  \setlength{\currentradius}{\layerRadius}
  \setlength{\currentradius}{.7\currentradius}
  \coordinate (layer0-center) at (135:\currentradius);
  \coordinate (layer0-left) at (0,0);
  \coordinate (layer0-right) at (0,0);
}
\NewDocumentCommand{\forsydeAtomSetLayer}{ >{ \SplitArgument { 1 } { : } } m } {%
  \forsydeAtomDrawLayer #1
  \setlength{\currentradius}{\currentradius+\layerRadius}%
  \pgfmathtruncatemacro{\currentLayer}{\currentLayer + 1}
}
\NewDocumentCommand{\forsydeAtomDrawLayer} {m m} {
  \def\myshift#1{\raisebox{\layerLabelShift}}
  \draw[name path=layer\currentLayer -rightpath,
      postaction={decorate, decoration={%
      text along path, reverse path, raise={\layerLabelShift},
      text align={align=right, right indent=\layerRightIndent},
      text={|\sffamily\LARGE|#1}}}]
  ([shift=(90:\currentradius)]0,0) coordinate (layer\currentLayer -right)
       arc (85:135:\currentradius) coordinate (layer\currentLayer -center);%
  \draw[name path=layer\currentLayer -leftpath,
      postaction={decorate, decoration={%
      text along path,
      text align={left,left indent=\layerLeftIndent}, reverse path, raise={1.5ex},
      text={|\sffamily\it|#2}}}]
  (layer\currentLayer -center) arc
      (135:180:\currentradius) coordinate  (layer\currentLayer -left);
}

\newcommand\forsydeAtomHighlightLayer[2][red]{
  \pgfmathtruncatemacro{\nextLayer}{#2 + 1}
  \gettikzy{(layer#2-right)}{\firstlayerrad}
  \gettikzy{(layer\nextLayer -right)}{\nextlayerrad}
  \coordinate (layer1r) at (layer#2-right);
  \begin{pgfonlayer}{background}
    \fill[#1]
    ([shift=(90:\firstlayerrad)]0,0) arc (85:180:\firstlayerrad)
    -- (layer\nextLayer -left) --  (layer\nextLayer -right) --
    (layer\nextLayer -left) arc (180:85:\nextlayerrad)
    --cycle;
  \end{pgfonlayer} 
}

\newsavebox{\forsydeAtomValue}
\newsavebox{\forsydeAtomExtValue}
\newsavebox{\forsydeAtomTagValue}
\newsavebox{\forsydeAtomTagExtValue}

\savebox{\forsydeAtomExtValue}{
  \begin{tikzpicture}[el/.style={shape=ellipse, draw, ultra thin},]
    \node [el, draw, inner sep=2pt] (v) {$v$};
    \node [anchor=south, yshift=-1pt] at (v.north) {$\bot$};
    \node [anchor=south, yshift=5pt,inner xsep=10pt] (e) at (v.north) {};
    \node [el, fit=(v)(e), inner sep=-2pt] (ve) {};   
  \end{tikzpicture}
}

\savebox{\forsydeAtomTagValue}{
  \begin{tikzpicture}[el/.style={shape=ellipse, draw, ultra thin},]
    \node [el, draw, inner sep=2pt] (v) {$v$};
    \node [anchor=south, yshift=-1pt] at (v.north) {$t$};
    \node [anchor=south, yshift=5pt,inner xsep=10pt] (e) at (v.north) {};
    \node [el, fit=(v)(e), inner sep=-2pt] (ve) {};   
  \end{tikzpicture}
}

\savebox{\forsydeAtomTagExtValue}{
  \begin{tikzpicture}[el/.style={shape=ellipse, draw, ultra thin},]
    \node [el, draw, inner sep=2pt] (v) {$v$};
    \node [anchor=south, yshift=-1pt] at (v.north) {$\bot$};
    \node [anchor=south, yshift=5pt,inner xsep=10pt] (e) at (v.north) {};
    \node [el, fit=(v)(e), inner sep=-2pt] (ve) {};   
    \node [anchor=west,inner sep=2pt] (d1) at (ve.east) {};
    \node [anchor=east,inner sep=2pt] (d2) at (ve.west) {};
    \node [anchor=south] (t) at (ve.north) {$t$};
    \node [el, fit=(ve)(t)(d1)(d2), inner sep=-5pt] {}; 
  \end{tikzpicture}
}

\savebox{\forsydeAtomValue}{
  \begin{tikzpicture}[el/.style={shape=ellipse, draw, ultra thin},]
  \node [el, inner sep=2pt] (value) {$v$};%
  \end{tikzpicture}
}

\endinput
%%% Local Variables:
%%% mode: latex
%%% TeX-master: "forsyde-pictures.sty"
%%% End:
